\documentclass[11pt,a4paper,spanish]{book}
\usepackage{estilo_unir-1}
\usepackage{apacite}

\usepackage{hyperref}
\numberwithin{equation}{chapter}
%\usepackage{caption}
\numberwithin{figure}{chapter}
%\usepackage{chngcntr}
%\counterwithin{equation}{chapter}
%\counterwithin{figure}{chapter}
%\counterwithin{table}{chapter}
%\renewcommand\theequation{\thechapter.\arabic{equation}}
%\renewcommand\thefigure{\thechapter.\arabic{figure}}
%\renewcommand\thetable{\thechapter.\arabic{table}}
%\renewcommand\theequation{\thechapter.\arabic{equation}}
%\counterwithin{figure}{chapter}
%\counterwithin{table}{chapter}
%\renewcommand\thefigure{\thechapter.\arabic{figure}}
%\renewcommand\thetable{\thechapter.\arabic{table}}
%\renewcommand\thefigure{\thechapter.\arabic{figure}}
%\makeatletter
%\renewcommand\p@figure{\thechapter..\arabic{figure}}
%\makeatother



%---------------------------
%título del trabajo y autor
%---------------------------
\title{Reconocimiento de patrones en sistemas industriales para la detección de fallos
		mediante algoritmos de Aprendizaje Automático}
\titulacion{Máster Universitario en Inteligencia Artificial}
\author{Jorge Elicer Lambraño Arroyo}
\date{11 de Septiembre de 2025}
\director{Gabriel Mauricio Ramírez Villegas}
\nombreciudad{Bogotá (Colombia)}

%---------------------------
%marges
%---------------------------
%\usepackage[margin=1.9cm]{geometry}
%---------------------------
%---------------------------
%---------------------------
%---------------------------
\begin{document}
\renewcommand{\listfigurename}{Índice de Ilustraciones}
\renewcommand{\listtablename}{Índice de Tablas}
\renewcommand{\contentsname}{Índice de Contenidos}
\renewcommand{\figurename}{Figura}
\renewcommand{\tablename}{Tabla} 

\maketitle

\frontmatter
\tableofcontents
\listoffigures
\listoftables

\chapter{Resumen}
% {\bf Nota:} En este apartado se introducirá un breve resumen en español del trabajo 
% realizado (extensión máxima: 150 palabras). Este resumen debe incluir el objetivo o 
% propósito de la investigación, la metodología, los resultados y las conclusiones.
El presente proyecto aborda la temática de  la detección de anomalías en rodamientos 
utilizando el CWRU Bearing Dataset como fuente de información. El objetivo principal 
consiste en comparar el desempeño de diversos modelos de aprendizaje automático con el 
fin de identificar cuál resulta más apropiado para el desarrollo de un detector de 
anomalías robusto, eficiente y aplicable en entornos industriales. Para ello, se 
implementaron diferentes etapas de preprocesamiento,  caracterización de las señales 
temporales y estrategías durante la fase el entrenamiento de los modelos, con el 
propósito de obtener un modelo optimizado con altas métricas de rendimiento que facilite
la tarea de detección de anomalías.Los resultados alcanzaron métricas superiores al 
90\% de exactitud, precisión, sensibilidad, F1-score y área bajo la curva RoC, lo que 
demuestra la viabilidad de aplicar machine learning en la detección temprana de fallas
en rodamientos

{\bf Palabras Clave:} Aprendizaje automático; CWRU Bearing Dataset; Detección de fallas;
Mantenimiento predictivo; Preprocesamiento de datos.

\chapter{Abstract}
% {\bf Nota:} En este apartado se introducirá un breve resumen en español del trabajo
% realizado (extensión máxima: 150 palabras). Este resumen debe incluir el objetivo o 
% propósito de la investigación, la metodología, los resultados y las conclusiones.
The present project addresses the topic of anomaly detection in bearings using the CWRU
Bearing Dataset as the source of information. The main objective is to compare the 
performance of various machine learning models in order to identify which one is the most
suitable for the development of a robust, efficient, and industrially applicable anomaly
detector. To this end, different preprocessing stages, time-series signal characterization,
and training strategies were implemented with the purpose of obtaining an optimized model
with high performance metrics that facilitates the anomaly detection task. The results 
achieved over 90\% in accuracy, precision, recall, F1-score and RoC curve, demonstrating
the feasibility of applying machine learning for early fault detection in bearings.

{\bf Palabras Clave:} Machine Learning; CWRU Bearing Dataset; Fault detectino;
Predictive maintenance; Data preprocessing.




\mainmatter
\chapter{Introducción}

% El primer capítulo es siempre una introducción. En la introducción se debe resumir 
% de forma esquemática % pero suficientemente clara lo esencial de cada una de las partes 
% del trabajo. La lectura de este primer capítulo ha de dar una idea clara de lo que se 
% pretendía, las conclusiones a las que se ha llegado y del procedimiento seguido.

% Como tal, es uno de los capítulos más importantes de la memoria. Las ideas principales
% a transmitir son la identificación del problema a tratar, la justificación de su 
% importancia, los objetivos generales a grandes rasgos y un adelanto de la contribución
% que esperas hacer.

% Típicamente una introducción tiene tres apartados:
% \begin{itemize}
% \item Motivación / justificación del tema a tratar
% \item Planteamiento del trabajo
% \item Estructura del trabajo
% \end{itemize}

En el presente trabajo se propone el desarrollo de un sistema de detección de anomalías
basado en técnicas de aprendizaje automático, aplicado al análisis de datos operativos
del sector industrial. A lo largo de este documento, el lector encontrará una 
contextualización para abordar el problema de la detección de anomalías, un estudio del
estado del arte en técnicas de detección de anomalías, un análisis detallado de los 
datos industriales utilizados, el diseño e implementación de diferentes modelos de 
aprendizaje automático, así como una comparación rigurosa de su rendimiento frente a 
soluciones existentes.


En el sector industrial, las anomalías se consideran como aquellos datos atípicos por 
fuera de los rangos operativos de los equipos, y en muchos casos están relacionadas con 
fallas, necesidad de mantenimiento o deficiencias en el funcionamiento. Usualmente, en 
el contexto industrial, los fallos están relacionados con valores atípicos. Antes que 
ocurra un fallo, suelen presentarse valores anómalos.  


La detección de estas anomalías, es una herramienta para la prevención de fallos. Una 
implementación de detección temprana de anomalías, puede prevenir fallos, reducir el 
riesgo de daños a la infraestructura y costos adicionales por reparaciones de equipos.
Como resultado, se obtienen como beneficios un aumento en la vida útil de los equipos 
gracias al mantenimiento preventivo, mejores condiciones de seguridad, optimización de
recursos y se reduce el riesgo de que los procesos productivos se detengan por fallos.  


Debido a la amplia gama de beneficios que ofrece la detección de anomalías para la 
prevención de fallos, ésta se ha convertido en un foco de estudio e investigación en 
muchos campos de la ingeniería, la estadística, la ciencia de datos y la informática.
En el presente trabajo se expone cómo se ha abordado este tema y se propone el 
desarrollo de una solución para detección de anomalías en operaciones industriales.
El enfoque de la solución está orientado a modelos estadísticos y de aprendizaje
automático. 


\section{Planteamiento del problema}

La detección de anomalías se ha convertido en un proceso clave, puesto que permite
identificar datos atípicos que podrían resultar en fallos o riesgos importantes en el
funcionamiento de muchas industrias. Las anomalías, aunque poco frecuentes, pueden 
representar eventos críticos que comprometen la integridad de procesos, generan 
pérdidas económicas, o incluso ponen en riesgo vidas humanas. 


Un alto porcentaje de las fallas que se presentan en maquinaria industrial pueden ser
detectadas gracias a anomalías que ocurren antes de éstas. Como resultado, una detección
temprana de anomalías permite la identificación de futuros fallos y realizar algún tipo 
de mantenimiento preventivo o acción correctiva con el fin de evitarlo. Como 
consecuencia, se obtiene un aumento de la vida útil de los equipos y disminución en el
número y costo de reparaciones. 


La detección de anomalías no se limita solamente al sector industrial. Existen casos de
uso donde se aplican conceptos de visión por computadora y deep learning para la 
detección de anomalías en imágenes médicas,\cite{zhou2021proxy}, \cite{guo2024encoder}, 
\cite{lu2024heterogeneous} \& \cite{zhong2022video}.
Del mismo modo, existen aplicaciones en el campo de la videovigilancia que combinan la 
visión por computadora y la detección de anomalías. \cite{zhong2022video}, 
\cite{zeng2021graph} \& \cite{zhang2022influence}. 
Estas soluciones identifican comportamientos inusuales en imágenes y videos que podrían
pasar desapercibidos ante la observación humana.


En los sectores financieros y de ciberseguridad, la detección de  anomalías es aplicada
para facilitar la tarea de detección de fraudes. Estos sistemas permiten identificar
transacciones inusuales o patrones de comportamiento sospechosos en tiempo real, lo que
resulta crucial para prevenir pérdidas económicas y violaciones de seguridad. Para este
caso de uso, los datos de entrada del sistema de detección son diferentes al anterior,
es decir, en lugar de requerir videos o imágenes como datos de entrada, se requiere
información de las transacciones. Por este gran espectro de aplicaciones la detección de
anomalías se ha convertido en una foco de estudio de la informática. Actualmente, se 
investigan modelos cada vez más precisos y escalables que puedan adaptarse a entornos 
dinámicos y con grandes volúmenes de datos. 


La gran diversidad de formatos que pueden tener los datos de entrada supone un reto para
la detección de anomalías. Cada uno de éstos requiere técnicas específicas de 
procesamiento, que pueden ser más o menos complejos dependiendo del tipo de formato. 
Algunos ejemplos de formatos de datos a los que se han diseñado detectores de anomalías 
son: tendencias y mediciones con respecto al tiempo, para el caso de variables 
industriales (presión, temperatura, humedad, flujo); señales de audio en el rango de 
frecuencia audible o fuera del rango audible humano; imágenes y videos en el caso de 
videoanalítica, y también textos, archivos y otros formatos no estructurados. 


A pesar de esa gran variedad, existen muchos algoritmos que permiten extraer 
características de los diferentes formatos de entrada. Esta etapa de extracción de 
características transforma los datos crudos en vectores o representaciones que 
pueden ser analizados de manera uniforme. Una vez obtenidas estas representaciones, 
es posible aplicar estrategias comunes de detección de anomalías, como técnicas 
estadísticas, modelos de aprendizaje automático supervisado o no supervisado, e 
incluso enfoques basados en aprendizaje profundo. Esto permite unificar el tratamiento 
de datos heterogéneos.


Además de su amplia variedad, los datos industriales suelen presentar características 
aumentan la complejidad de su procesamiento, como por ejemplo, ser multivariantes, tener
una relación señal a ruido desfavorable, contener un alto porcentaje de valores 
faltantes o nulos, estar organizados en series temporales, entre otras. Esto plantea un 
desafío técnico para un sistema de monitoreo tradicional. En este contexto, surge la 
siguiente necesidad: ¿Cómo se puede construir un sistema de detección de anomalías 
altamente efectivo, preciso y flexible para poder adaptarse a situaciones complejas y 
que permita identificar de manera confiable comportamientos atípicos en entornos 
industriales reales?. 


El presente trabajo se plantea como una respuesta a esta necesidad, mediante el 
desarrollo de una solución de detección de anomalías, orientada al análisis de variables
industriales. La solución buscará alcanzar un desempeño comparable o superior al de las
técnicas existentes en el estado del arte. 


\section{Motivación}

Este proyecto surge de la necesidad de contar con soluciones inteligentes capaces de 
prevenir eventos críticos mediante la detección de anomalías, el análisis de datos y el
aprendizaje automático. 
La  importancia de la detección temprana de anomalías como solución ante este problema 
radica en que permite realizar acciones correctivas que eviten que se generen fallas en 
los sistemas industriales, puesto que los fallos normalmente van precedidos de 
comportamientos anómalos en las variables que se están censando.


Tomar acciones correctivas antes que se presenten fallos tienen múltiples beneficios,
puesto que los fallos pueden ocasionar que procesos industriales se detengan generando
incumplimientos en la producción. Los efectos de los fallos son todavía más graves para
aquellos procesos que funcionan continuamente, las interrupciones se convierten en 
grandes pérdidas económicas e incumplimiento de las metas de producción. 
De esta manera, la prevención de fallos supone grandes ahorros a las industrias y reduce
el riesgo de que se generen interrupciones. De esta manera, por medio de la detección
temprana, se mejora la confiabilidad operativa y se contribuye a la reducción de costos
asociados con paradas no planificadas, daños a equipos y pérdidas en la producción.


La detección de anomalías puede integrarse con otras tecnologías como los agentes 
inteligentes. 
La integración con éstos tiene un mayor impacto en la eficiencia de los sistemas 
industriales. Esta integración permite que se acepte o se rechace los resultados de 
los detectores de anomalías y que se tomen las diferentes decisiones 
\cite{jidiga2014anomaly}.


La necesidad de soluciones cada vez más automatizadas, con constante monitoreo y que 
generen una gran cantidad de datos, hace que la detección de anomalías sea esencial 
para garantizar la seguridad, confiabilidad y eficiencia de sistemas industriales y en 
otras áreas. 
La capacidad de identificar estos eventos de manera oportuna no solo permite prevenir 
fallos o fraudes, sino que también habilita nuevos modelos de mantenimiento inteligente 
y optimización de recursos y beneficios a largo plazo. Investigar y perfeccionar 
técnicas de detección de anomalías representa un reto de gran relevancia para el 
desarrollo de múltiples sectores industriales.


\section{Plantamiento del trabajo}

Tomando como punto de partida la necesidad de identificar anomalías en entornos 
industriales con el fin de prevenir fallos y optimizar procesos, este trabajo propone 
el desarrollo de una solución basada en aprendizaje automático que permita detectar 
comportamientos atípicos a partir del análisis de variables operativas. El objetivo es 
modelar el comportamiento normal del sistema y reconocer, de manera oportuna y precisa, 
desviaciones significativas que puedan indicar fallas incipientes. 


Para ello, se implementarán estrategias basadas en aprendizaje automático, que permitan 
modelar el comportamiento normal del sistema y reconocer desviaciones significativas de 
manera oportuna y precisa. 
El desarrollo de esta solución requiere múltiples fases. 
El primer paso es el análisis exploratorio de los datos de las variables de entrada con 
el fin de entender la estructura,  calidad y distribución de las variables que 
alimentarán el modelo. 
En esta etapa se realiza una evaluación de la calidad de los datos, donde se mide el 
porcentaje de datos faltantes, duplicados y valores atípicos. 
Adicionalmente, se calculan los estadísticos más importantes y las correlaciones entre 
las variables. 
También se definen las transformaciones de las variables y todas las operaciones que 
hacen parte del preprocesamiento de los datos.


Basándose en los formatos de los datos de entrada y salida se construirá un modelo 
de aprendizaje de máquina. 
El modelo debe ser capaz de procesar los datos en el formato de entrada 
(por ejemplo, series temporales, datos multivariantes) y generar una salida que permita
identificar si los valores de entrada corresponden a una anomalía o no. 
La solución será evaluada bajo criterios de precisión, sensibilidad, especificidad y 
eficiencia computacional.


Se espera que el modelo resultante de esta fase genere resultados con una calidad igual 
o superior a los modelos que hoy en día se encuentran en estado del  arte. Con el fin de
hacer una comparación entre los modelos del estado del arte y el modelo propuesto, se 
miden métricas de rendimiento bajo condiciones experimentales homogéneas. 
Asimismo, se discutirán las ventajas, limitaciones y oportunidades de mejora de la 
solución propuesta, considerando su potencial implementación en entornos industriales
reales.


\section{Estructura de la memoria}


El presente trabajo tiene la siguiente estructura:


\textbf{Capítulo 1:} Se brinda una breve introducción del presente trabajo. 
Se expone la motivación que ha impulsado este desarrollo y por qué es importante 
profundizar en la detección de anomalías. 
Además, se plantea con claridad el problema de investigación, formulado en términos 
precisos. 
Finalmente, se describe la estructura del presente documento, explicando de forma clara 
cómo está dividido el contenido del estudio y qué se espera encontrar en cada capítulo.


\textbf{Capítulo 2:} Se realiza una revisión exhaustiva del estado del arte, también se 
realiza un barrido de las diferentes técnicas que están relacionadas con la detección 
de anomalías. 
En ella se examinan los enfoques más relevantes utilizados en la literatura, abarcando 
desde una perspectiva tradicional así como enfoques más modernos basados en inteligencia 
artificial y aprendizaje automático. 
Asimismo, se analizan los resultados que han obtenido otros investigadores en este campo, 
junto con las estrategias y métodos implementados.


\textbf{Capítulo 3:} Se presentan los objetivos generales y específicos que guían cada 
etapa de desarrollo del presente estudio. 
El objetivo general enuncia de manera precisa la finalidad del estudio y su impacto 
esperado. 
Por otro lado, los objetivos específicos desglosan dicho propósito en metas alcanzables 
y medibles que estructuran el camino metodológico a seguir. 
Esta sección es fundamental, ya que permite al lector entender qué se espera lograr y 
cómo se evaluará el cumplimiento de las metas.


\textbf{Capítulo 4:} Se describe detalladamente la metodología adoptada para llevar a 
cabo la investigación. 
Se explican los pasos y procedimientos que se seguirán, desde la recolección de los 
datos y su procesamiento, hasta la implementación de los modelos y técnicas 
seleccionadas para la detección de anomalías. 
Esta sección incluye el conjunto de herramientas y tecnologías utilizadas, así como los 
criterios de validación y evaluación de los modelos a implementar. 
La metodología constituye el puente entre los objetivos planteados y los resultados que
se desean obtener.


\textbf{Capítulo 5:} Se presentan los resultados obtenidos tras la aplicación de la 
metodología previamente descrita. 
Se analizan de forma crítica y detallada los datos generados, utilizando 
representaciones visuales y métricas pertinentes que faciliten su comprensión. 
Además, se comparan los resultados con estudios previos y se realizan las respectivas 
interpretaciones, lo que permite identificar patrones, tendencias y posibles 
explicaciones para los hallazgos obtenidos. 


\textbf{Capítulo 6:} Se presentan conclusiones derivadas del estudio, las cuales 
sintetizan los principales hallazgos y resultados obtenidos y su relevancia en el 
contexto del problema investigado. 
Se reflexiona sobre el cumplimiento de los objetivos y la metodología planteada. 
Además, se identifican las oportunidades de mejora, reconociendo aquellos aspectos 
que podrían explorarse con mayor profundidad, y dan espacio a diversas líneas de 
investigación futura que permitan extender y enriquecer el presente trabajo.


\chapter{Contexto y Estado del Arte}

\chapter{Identificación de Requisitos}


\chapter{Objetivos}

Dada la creciente necesidad de identificar comportamientos atípicos en la industria, 
se plantean una serie de objetivos claros que permiten abordar el problema desde una 
perspectiva integral. 
Estos objetivos guían el diseño, implementación y evaluación de modelos capaces de 
detectar desviaciones anomalías de manera eficiente y precisa.


\section{Objetivo General}

\textbf{Desarrollar} una solución basada en algoritmos de aprendizaje automático 
capaz de detectar anomalías en datos provenientes de entornos industriales, garantizando
un desempeño medible a través de métricas estándares.
La solución deberá alcanzar métricas de rendimiento superiores a 0.85 en métricas como
exactitud y F1, demostrando alta eficacia en la identificación de anomalías.

\section{Objetivos Específicos}

\begin{itemize}

\item Desarrollar la arquitectura de una solución para la detección de anomalías, donde 
se definan claramente los distintos componentes que la integran, sus funciones y el 
flujo de trabajo que sigue la solución, con el fin de  garantizar una implementación 
clara y coherente.


\item Analizar conjuntos de datos industriales, con el fin de garantizar que cumpla 
con los requerimientos necesarios para la implementación de algoritmos de aprendizaje 
automático y realizar los ajustes y transformaciones a los datos en caso de ser necesario. 


\item Seleccionar los modelos y algoritmos de aprendizaje automático que se integrarán
en la solución, e implementar los ajustes necesarios en sus parámetros y configuración
con el fin de optimizar su rendimiento en términos de precisión, sensibilidad y otras
métricas relevantes.


\item Evaluar el desempeño de los modelos seleccionados, utilizando un conjunto de 
condiciones experimentales homogéneas que garanticen una evaluación objetiva, 
identificando fortalezas, limitaciones y oportunidades de mejora.


\item Implementar el análisis del conjunto de datos, la selección, entrenamiento y 
evaluación de modelos, la ejecución de pruebas controladas y la disponibilización de la 
solución final, utilizando herramientas de desarrollo especializadas. 

\end{itemize}



\chapter{Metodología de trabajo}


\chapter{Desarrollo del trabajo}

\chapter{Conclusiones y Trabajo Futuro}

En el presente  trabajo se alcanzaron métricas superiores al 90\% en todos los modelos
evaluados, superando el resultado esperado, lo cual refleja la solidez del enfoque 
planteado para la detección de anomalías en entornos industriales. La etapa de 
preprocesamiento de los datos fue un paso clave: la caracterización de las series de 
tiempo, junto con un adecuado balanceo del dataset y la búsqueda de hiperparámetros, 
permitió que los modelos pudieran generalizar mejor y mostraran robustez frente al 
sobreajuste. Estos resultados confirman la efectividad del pipeline desarrollado para 
transformar datos crudos en características discriminativas útiles para los modelos 
de predicción.


Debido al alto rendimiento alcanzado, todos los modelos evaluados podrían aplicarse en 
una solución in-situ. No obstante, los que obtuvieron las mejores métricas fueron 
XGBoost y LightGBM, confirmando su eficacia como algoritmos de referencia en problemas
de clasificación complejos, y un buen referente para la detección de anomalías. Sin 
embargo, si la implementación requiere simplicidad, interpretabilidad y bajo costo 
computacional, los modelos clásicos como la regresión logística, los árboles de decisión
o Naive Bayes resultan altamente recomendables, ya que ofrecen un equilibrio entre 
eficiencia, precisión e interpretabilidad.


Finalmente, gracias al uso de modelos con mayor interpretabilidad, se identificó que 
variables como los valores pico (maximum), la desviación estándar (std), el valor RMS,
la frecuencia fundamental (f0) y la distorsión armónica total (THD) son las que más 
aportan a la clasificación de una observación como anomalía. Este hallazgo  además de 
validar la calidad del modelo, proporciona información valiosa para la toma de 
decisiones en el ámbito industrial, al permitir inferir cuáles son los factores físicos
más determinantes en el diagnóstico temprano de fallos.


Como línea de trabajo futuro, una posibilidad interesante es la integración de los 
modelos desarrollados en un PLC (Controlador Lógico Programable) u otro dispositivo
de cómputo industrial. Esto permitiría que la detección de anomalías se ejecute de 
manera automática y en tiempo real dentro del propio entorno productivo. La adaptación
de los algoritmos a este tipo de hardware representa un reto, especialmente en lo 
relacionado con la optimización del uso de recursos computacionales, pero a la vez abre
la puerta a aplicaciones industriales de gran impacto.


Otra línea de investigación a considerar es el diseño de un algoritmo de clasificación
que no solo detecte anomalías, sino que también identifique el tipo específico de falla:
un clasificador multiclase.  Este enfoque aportaría un valor agregado, ya que permitiría
pasar de un sistema de simple diagnóstico (anómalo/no anómalo) a un sistema de prognosis
y clasificación de fallas, capaz de diferenciar entre problemas la gravedad del fallo y 
el lugar donde se presenta. Con ello, las soluciones podrían evolucionar hacia un 
mantenimiento predictivo más preciso y especializado.


En la Ecuación \eqref{eq:eq1secCTF}


\begin{equation}\label{eq:eq1secCTF}
M=\begin{pmatrix}
	m_{11}&m_{12}\\
	m_{21}&m_{22}
\end{pmatrix}
\end{equation}

En la siguiente Tabla \ref{tab:tab1secCTF}

\begin{table}[h]
\centering
\begin{tabular}{|c|c|}
	\hline
	1 & 2 \\
	\hline
	22 & 11 \\
	\hline
\end{tabular}
\caption{Tabla 1}
\label{tab:tab1secCTF}
\end{table}

En la siguiente Figura \ref{fig:fig1secCTF}

\begin{figure}[h]
\includegraphics[width= 0.8\textwidth]{logo_unir}
\caption{Logo Unir}
\label{fig:fig1secCTF}
\end{figure}

\cite{PIMENTEL2016744} \cite{da_S_Bessa_2023}

%\begin{thebibliography}{a}
%\bibitem{etiqueta} \textsc{Autores},
%\textit{nombre referencia.}
%Información addicional
%\end{thebibliography}
\bibliographystyle{apacite}
\bibliography{bibliografia}

\appendix
\chapter{Apendices}

\end{document}





















