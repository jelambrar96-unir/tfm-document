\documentclass[11pt,a4paper,spanish]{book}
\usepackage{estilo_unir-1}
\usepackage{apacite}

\usepackage{hyperref}
\numberwithin{equation}{chapter}
%\usepackage{caption}
\numberwithin{figure}{chapter}
%\usepackage{chngcntr}
%\counterwithin{equation}{chapter}
%\counterwithin{figure}{chapter}
%\counterwithin{table}{chapter}
%\renewcommand\theequation{\thechapter.\arabic{equation}}
%\renewcommand\thefigure{\thechapter.\arabic{figure}}
%\renewcommand\thetable{\thechapter.\arabic{table}}
%\renewcommand\theequation{\thechapter.\arabic{equation}}
%\counterwithin{figure}{chapter}
%\counterwithin{table}{chapter}
%\renewcommand\thefigure{\thechapter.\arabic{figure}}
%\renewcommand\thetable{\thechapter.\arabic{table}}
%\renewcommand\thefigure{\thechapter.\arabic{figure}}
%\makeatletter
%\renewcommand\p@figure{\thechapter..\arabic{figure}}
%\makeatother



%---------------------------
%título del trabajo y autor
%---------------------------
\title{Reconocimiento de patrones en sistemas industriales para la detección de fallos
		mediante algoritmos de Aprendizaje Automático}
\titulacion{Máster Universitario en Inteligencia Artificial}
\author{Jorge Elicer Lambraño Arroyo}
\date{11 de Septiembre de 2025}
\director{ }
\nombreciudad{Bogotá (Colombia)}

%---------------------------
%marges
%---------------------------
%\usepackage[margin=1.9cm]{geometry}
%---------------------------
%---------------------------
%---------------------------
%---------------------------
\begin{document}
\renewcommand{\listfigurename}{Índice de Ilustraciones}
\renewcommand{\listtablename}{Índice de Tablas}
\renewcommand{\contentsname}{Índice de Contenidos}
\renewcommand{\figurename}{Figura}
\renewcommand{\tablename}{Tabla} 

\maketitle

\frontmatter
\tableofcontents
\listoffigures
\listoftables

\chapter{Resumen}
% {\bf Nota:} En este apartado se introducirá un breve resumen en español del trabajo 
% realizado (extensión máxima: 150 palabras). Este resumen debe incluir el objetivo o 
% propósito de la investigación, la metodología, los resultados y las conclusiones.
El presente proyecto aborda la temática de  la detección de anomalías en rodamientos 
utilizando el CWRU Bearing Dataset como fuente de información. El objetivo principal 
consiste en comparar el desempeño de diversos modelos de aprendizaje automático con el 
fin de identificar cuál resulta más apropiado para el desarrollo de un detector de 
anomalías robusto, eficiente y aplicable en entornos industriales. Para ello, se 
implementaron diferentes etapas de preprocesamiento,  caracterización de las señales 
temporales y estrategías durante la fase el entrenamiento de los modelos, con el 
propósito de obtener un modelo optimizado con altas métricas de rendimiento que facilite
la tarea de detección de anomalías.Los resultados alcanzaron métricas superiores al 
90\% de exactitud, precisión, sensibilidad, F1-score y área bajo la curva RoC, lo que 
demuestra la viabilidad de aplicar machine learning en la detección temprana de fallas
en rodamientos

{\bf Palabras Clave:} Aprendizaje automático; CWRU Bearing Dataset; Detección de fallas;
Mantenimiento predictivo; Preprocesamiento de datos.

\chapter{Abstract}
% {\bf Nota:} En este apartado se introducirá un breve resumen en español del trabajo
% realizado (extensión máxima: 150 palabras). Este resumen debe incluir el objetivo o 
% propósito de la investigación, la metodología, los resultados y las conclusiones.
The present project addresses the topic of anomaly detection in bearings using the CWRU
Bearing Dataset as the source of information. The main objective is to compare the 
performance of various machine learning models in order to identify which one is the most
suitable for the development of a robust, efficient, and industrially applicable anomaly
detector. To this end, different preprocessing stages, time-series signal characterization,
and training strategies were implemented with the purpose of obtaining an optimized model
with high performance metrics that facilitates the anomaly detection task. The results 
achieved over 90\% in accuracy, precision, recall, F1-score and RoC curve, demonstrating
the feasibility of applying machine learning for early fault detection in bearings.

{\bf Palabras Clave:} Machine Learning; CWRU Bearing Dataset; Fault detectino;
Predictive maintenance; Data preprocessing.




\mainmatter
\chapter{Introducción}

El primer capítulo es siempre una introducción. En la introducción se debe resumir de forma esquemática pero suficientemente clara lo esencial de cada una de las partes del trabajo. La lectura de este primer capítulo ha de dar una idea clara de lo que se pretendía, las conclusiones a las que se ha llegado y del procedimiento seguido.

Como tal, es uno de los capítulos más importantes de la memoria. Las ideas principales a transmitir son la identificación del problema a tratar, la justificación de su importancia, los objetivos generales a grandes rasgos y un adelanto de la contribución que esperas hacer.

Típicamente una introducción tiene tres apartados:
\begin{itemize}
\item Motivación / justificación del tema a tratar
\item Planteamiento del trabajo
\item Estructura del trabajo
\end{itemize}


El primer capítulo es siempre una introducción. En la introducción se debe resumir de forma esquemática pero suficientemente clara lo esencial de cada una de las partes del trabajo. La lectura de este primer capítulo ha de dar una idea clara de lo que se pretendía, las conclusiones a las que se ha llegado y del procedimiento seguido.

Como tal, es uno de los capítulos más importantes de la memoria. Las ideas principales a transmitir son la identificación del problema a tratar, la justificación de su importancia, los objetivos generales a grandes rasgos y un adelanto de la contribución que esperas hacer.





%hishidhdihihsihdish

El primer capítulo es siempre una introducción. En la introducción se debe resumir de forma esquemática pero suficientemente clara lo esencial de cada una de las partes del trabajo. La lectura de este primer capítulo ha de dar una idea clara de lo que se pretendía, las conclusiones a las que se ha llegado y del procedimiento seguido.

Como tal, es uno de los capítulos más importantes de la memoria. Las ideas principales a transmitir son la identificación del problema a tratar, la justificación de su importancia, los objetivos generales a grandes rasgos y un adelanto de la contribución que esperas hacer.



El primer capítulo es siempre una introducción. En la introducción se debe resumir de forma esquemática pero suficientemente clara lo esencial de cada una de las partes del trabajo. La lectura de este primer capítulo ha de dar una idea clara de lo que se pretendía, las conclusiones a las que se ha llegado y del procedimiento seguido.

Como tal, es uno de los capítulos más importantes de la memoria. Las ideas principales a transmitir son la identificación del problema a tratar, la justificación de su importancia, los objetivos generales a grandes rasgos y un adelanto de la contribución que esperas hacer.



El primer capítulo es siempre una introducción. En la introducción se debe resumir de forma esquemática pero suficientemente clara lo esencial de cada una de las partes del trabajo. La lectura de este primer capítulo ha de dar una idea clara de lo que se pretendía, las conclusiones a las que se ha llegado y del procedimiento seguido.

Como tal, es uno de los capítulos más importantes de la memoria. Las ideas principales a transmitir son la identificación del problema a tratar, la justificación de su importancia, los objetivos generales a grandes rasgos y un adelanto de la contribución que esperas hacer.



\chapter{Contexto y Estado del Arte}

\chapter{Identificación de Requisitos}

\chapter{Objetivos}

\chapter{Desarrollo del trabajo}

\chapter{Conclusiones y Trabajo Futuro}

En la Ecuación \eqref{eq:eq1secCTF}


\begin{equation}\label{eq:eq1secCTF}
M=\begin{pmatrix}
	m_{11}&m_{12}\\
	m_{21}&m_{22}
\end{pmatrix}
\end{equation}

En la siguiente Tabla \ref{tab:tab1secCTF}

\begin{table}[h]
\centering
\begin{tabular}{|c|c|}
	\hline
	1 & 2 \\
	\hline
	22 & 11 \\
	\hline
\end{tabular}
\caption{Tabla 1}
\label{tab:tab1secCTF}
\end{table}

En la siguiente Figura \ref{fig:fig1secCTF}

\begin{figure}[h]
\includegraphics[width= 0.8\textwidth]{logo_unir}
\caption{Logo Unir}
\label{fig:fig1secCTF}
\end{figure}

\cite{PIMENTEL2016744} \cite{da_S_Bessa_2023}

%\begin{thebibliography}{a}
%\bibitem{etiqueta} \textsc{Autores},
%\textit{nombre referencia.}
%Información addicional
%\end{thebibliography}
\bibliographystyle{apacite}
\bibliography{bibliografia}

\appendix
\chapter{Apendices}

\end{document}





















